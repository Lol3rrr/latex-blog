\documentclass[a4paper, 11pt]{article}

\title{My Spin on MVCC for a toy database}
\date{October 2023}

\begin{document}
\section{Introduction}
MVCC, or Multi Version Concurrency Control, describes the group of methods used by databases to support multiple
users accessing and using the database at the same time (concurrently). \\
As far as I am aware most databases implement their custom mechanisms for that, which fit the goals and constraints
of said project, however actually figuring out how these things work and the different approaches can seem daunting. \\
So in this I want to summarize and think about an idea I had for an MVCC approach that I could use in my WIP database
that is intended for learning and playing around with.

\section{Properties}
These are the core properties that I was interested in, from a standard database point of view
\begin{enumerate}
\item Repetable Reads
\item (Snapshot isolation)
\item (Serializable)
\end{enumerate} \\
However there were some more interesting aspects from a technical point of view as well
\begin{enumerate}
\item Transactions should be as independent as possible
\item Transactiosn should not block each other
\item High scalability
\end{enumerate}

\section{Idea}
The Basic idea is to store version/transaction-ids and for reads you only consider the versions that were committed at
the time of starting your transaction and for writes you obtain the next highest unused transaction id and perform all
your writes during the transaction with that id and at the end you try to commit it, by "publishing" your transaction-id
as being committed or aborted.

\begin{enumerate}
\item $C \subseteq \mathbb{N}$: The Committed Transactions
\item $B \subseteq \mathbb{N}$: The Aborted Transactions
\item $C \cap B = \emptyset$
\item $WT \in \mathbb{N}$
\end{enumerate}

\subsection{Starting a Transaction}
Get a Snapshot of $C_t = C$ and $B_t = B$, these won't change for the entire duration of the Transaction and are what
will ensure the "Repetable Read" property.

\subsection{Performing a Read}
When performing a Read, you search for the most recent row entry where the commit transaction id is the highest transaction id
that is in $C$, so $tx_{row} \in C$

\subsection{Performing a Write}
For the first write, you obtain a $WT_t = WT + 1$ and update $WT = WT + 1$. \\
Then when performing any write, you simply write the row entry the transaction id set to $WT_t$

\subsection{Committing}
% TODO

\subsection{Aborting}
% TODO

\end{document}
