\documentclass[a4paper, 11pt]{article}

\title{Interval-Stack (or something?)}
\date{October 2023}

\begin{document}

\section{Why?}
Recently I needed a way to store a set of timestamps, where I know that you should usually have all timestamps
up to some point and maybe a couple of gaps, but mainly one continuos sequence. \\
Additionally I needed to be able to get the largest timestamp so that $1..n$ would be in the set of timestamps
I am storing. Naturally I found things like Range-Trees or Interval-Trees, however that seemed like overkill in
this case, as I knew I would be mostly dealing with kind of increasing numbers and mostly one big interval with
maybe a couple of smaller gaps, especially regarind the most recent timestamps. \\
\\
This lack of finding something that I felt did the job well and was not gonna be overly complex or just like too
big of a tool for the job (does that make sense?) lead me to think of a very basic data structure that solves this
problem very nice, which I am gonna call Interval-Stacks.

\section{What is an 'Interval-Stack' then?}
Well I am glad you asked. \\
The Idea is not too unsimilar from the previously mentioned range/interval trees, however we are not gonna store
it as a tree, at least not directly as there probably is some tree-ish structure hiding there somewhere. \\


\section{How?}

\section{Summary}
\subsection{Use-Cases}

\end{document}
