\documentclass[a4paper, 11pt]{article}

\title{Interval-Stack (or something?)}
\date{October 2023}

\begin{document}

\section{Why?}
Recently I needed a way to store a set of timestamps, where I know that you should usually have all timestamps
up to some point and maybe a couple of gaps, but mainly one continuos sequence. \\
Additionally I needed to be able to get the largest timestamp so that $1..n$ would be in the set of timestamps
I am storing. Naturally I found things like Range-Trees or Interval-Trees, however that seemed like overkill in
this case, as I knew I would be mostly dealing with kind of increasing numbers and mostly one big interval with
maybe a couple of smaller gaps, especially regarind the most recent timestamps. \\
\\
This lack of finding something that I felt did the job well and was not gonna be overly complex or just like too
big of a tool for the job (does that make sense?) lead me to think of a very basic data structure that solves this
problem very nice, which I am gonna call Interval-Stacks.

\section{What is an 'Interval-Stack' then?}
Well I am glad you asked. \\
The Idea is not too unsimilar from the previously mentioned range/interval trees, however we are not gonna store
it as a tree, at least not directly as there probably is some tree-ish structure hiding there somewhere. \\


\section{How?}
\subsection{Description}
Instead of storing the individual timestamps, we only store sorted ranges/intervals of timestamps in a stack-like
datastructure, where we can also merge/remove elements anywhere in the "Stack". \\
Then every time we insert a new timestamp, we check if it already falls into one of the stored intervals, then
we don't have to do anything, otherwise we search for the position in the stack, where the timestamp belongs and
insert it there as an interval with a length of 1. \\
After inserting we check the stack again and look for any intervals can be merged, if so we merge them into a single
interval, reducing/keeping the size of the stack the same. \\
\\
This structure means that the smallest interval is the lowest in the stack, providing constant access time to it,
likewise in the best case we only have to store a single interval on the stack, as it is just a single continuos
interval of timestamps and in the worst case you have to store all $n$ timestamps on the stack, just like a normal
set, while stilling having quick access to the lowest one.

\subsection{Pseudo-Code}
\begin{verbatim}
fn contains(stack, timestamp) {
	for interval in stack {
		if timestamp in interval {
			return true;
		}
	}

	return false;
}

fn insert_pos(stack, timestamp) {
	for (idx, interval) in stack {
		if interval < timestamp {
			return idx;
		}
	}

	return 0;
}

fn merge(stack) {
	% TODO
}

fn insert(stack, timestamp) {
	if contains(stack, timestamp) {
		return;
	}

	pos = insert_pos(stack, timestamp);

	insert(stack, pos, timestmp);

	merge(stack);
}
\end{verbatim}

\section{Summary}
\subsection{Use-Cases}


\end{document}
